\documentclass{article}
\usepackage{graphicx}
\usepackage{bm}

\begin{document}

\title{COMP 379 Assignment 1}
\author{Mark Griffith - 1422270}

\maketitle

\section{Objective}
This project was an oppurtunity to to practice UNIX process managment function calls
in order to implement a simple and functional UNIX shell.

\section{Design Overview}
\textbf{Program File:} a1shell.c \\
\textbf{Executable:} a1shell \\
\textbf{Function Returns:} 0 on success and -1 on failure. \\
\textbf{Makefile Options}: make $|$ make clean $|$ make tar \\

\noindent
\textbf{usage()}: Prints the proper syntax for calling the a1shell executable. \\

\noindent
\textbf{main()}: First, main() limits the CPU time of the program to avoid a hanging
process.
Next, a fork() issues a child process (called a1monitor). The parent process is the
shell (called a1shell). \\
\noindent
Note: main() uses the return call to exit the program anytime BEFORE the processes
are created. This is purely a style choice and functions the same as \_exit(). \\

\noindent
The two processes created in main() have the following design.
\begin{enumerate}
  \item \textbf{a1shell}: this is the parent process of the main program
   \begin{itemize}
     \item Scans for user input from stdout
     \item Passes the user command to the parse function. Supported commands include...\\

     \textbf{cd}: uses chdir() system call to change user directory \\
     \textbf{pwd}: prints working directory to stdout with getpwd\\
     \textbf{umask}: prints current mask as well as S\_IRWXG, S\_IRWXO, and S\_IRWXU\\
     \textbf{done}: terminates the a1shell process via the \_exit() system call \\
     \textbf{cmd arg1 arg2 ...} : forks one child process which uses execl() to execute
     "/bin/bash cmd arg1 arg2 ...". The parent process waits for the child to terminate
     and prints the command execution time of the parent and child processes to
     stdout. \\

    \noindent
    In order to make my program more modular, I have created functions
    for the seperate functionalities of the a1shell process as well as
    a parser for these functions. These functions include... \\
    \textbf{parse()}, \textbf{change\_dir()}, \textbf{print\_dir()}, \textbf{print\_mask()},
    and \textbf{execute\_bash()}. \\

     \item Exits successfully with \_exit(EXIT\_SUCCESSFUL) and exits unsuccessfully
     with \_exit(EXIT\_UNSECCESSFULL)
   \end{itemize}

     Note: as per the specifications in the assignment outline, I have designed the a1shell
     process to terminate without waiting on its child process. The child process (a1monitor) has
     been designed to terminate once its parent (a1shell) has terminated (see a1monitor design
     below). \\
   \item \textbf{a1monitor}: This is the child process of the a1shell process
   \begin{itemize}
     \item Is in a while loop as long as its parent pid is the pid of a1shell
     \item Designed to terminate AFTER its parent process terminates.
     \item Prints the current time and date, followed by the information found
     in /sys/loadavg to stdout
     \item Always exits successfully using \_Exit(EXIT\_SUCCESFUL)
   \end{itemize}
\end{enumerate}

\section{Project Status}
This project currently implements all functionalities outlined in the
assignment outline. The main difficulities in the implementation of this
assignment came from finding legitimate ways of testing the several
functionalites of the code. Fortunately, I found that testing gave me the
oppurtunity to learn and better my understanding of the various system calls
necessary for the completion of this assignment.


\section{Testing and Results}
\textbf{a1monitor Testing}: The only testing to be done a1monitor was asserting that
it's output was correct and that it exited properly after the a1shell process
terminated. This was done by terminating the a1shell process and running
\verb=ps -al | grep a1shell= to ensure the a1monitor process was not still running. \\

\noindent
\textbf{a1shell Testing}: Testing each of a1shell's supported commands (cd, pwd,
umask, done) was done manually while running the process. My main testing strategy
here was to imitate the behaviour of the bash shell when running identical commands. \\

\noindent
\textbf{Timing Test}: Testing the execl() call was a little trickier because I
had to make sure I was timing the process correctly. \\
\noindent
First, to test the total time elapsed, I ran a sleep command
from a1shell. The expected output for total time was the number of seconds I chose
to sleep for. i.e. sleep 10 would yield 'total real time: 10.00'. \\
To test the timing of the parent (a1shell) all I had to do was add a sleep anywhere
in it's code and that it's user and cpu time added up to (roughly) the number of slept
seconds. \\
This could not be done for the child process because I could not add a sleep inside
the execl() system call. Therefore, I made the child process do some work before
running execl() in the form of a for loop. i.e.
\begin{verbatim}
  for(i = 0; i<100000; i++) {
    printf("Hello\n");
  }
\end{verbatim}
This yielded
\begin{verbatim}
  execl: total user time: 0.02
  execl: total cpu time: 0.17
\end{verbatim}
so I knew the timing was working for both the parent and child processes. \\

\noindent
Similarly, I used this strategy to test if setrlimit() was doing its job.
I set a 1-second limit on the CPU time and ran a for loop that lasted longer than
a minute and the program exited, as expected. \\
\section{Acknowledgments}
https://linux.die.net

\end{document}












