\documentclass{article}
\usepackage{graphicx}
\usepackage{bm}

\begin{document}

\title{CMPU 379 Assignment 2 - Phase 1 Answers}
\author{Mark Griffith - 1422270}

\maketitle

\noindent
\textbf{1.} No, you cannot simply monitor the size
of a named pipe (FIFO) on the file system. Using
ls -lh or du -sh will show the file size as 0B,
even as it is being continually written to by
a process. This is because data written to a fifo is
passed through the kernel without writing to the filesystem.
One method of monitoring the traffic between the
two processes would be to trace one of them (see
strace or ptrace). \\

\noindent
\textbf{2.} No. For processes to communicate via
pipes, they must be on the same host. This
is because pipes only exist within a single
host and use buffers within virtual files
instead of packets. To accomplish inter
process communication between different hosts,
you need to use sockets. \\

\noindent
\textbf{3.} When B attempts to open the FIFO,
it is not blocked. The open (for B) would be blocked if the
file were not opened for write by process A.\\

\noindent
\textbf{4.} Yes. Process B can detect that
process A is locked while A is waiting for
user input. B can use 'lockf(fd, F\_TEST, bytes)'
to test if there is a lock on the FIFO. \\

\noindent
\textbf{5.}
Loop (b) works properly. It first clears the buffer (sets all
elements to 0), reads from stdin, and writes to stdout. \\

\noindent
The reason loop (a) doesn't work properly is because when the
buffer is initialized it is not 0-set. Whatever was
previously in the buffer will still be there. Strings are
'0' terminated and so when printing the string in the
buffer, it will print until the '0'. If we write
"watermelon" to the buffer on the first iteration (of loop a),
and "apple"
on the second iteration, we will see the output "applemelon". \\

\noindent
* Testing of each situation above was done by re-creating the
situations on the lab machine.
\end{document}
