\documentclass{article}
\usepackage{graphicx}
\usepackage{bm}

\begin{document}

\title{COMP 372 Assignment 1}
\author{Mark Griffith}

\maketitle

\section{Objectives}
This project was an oppurtunity for myself to practice UNIX process managment function calls
in order to implement a simple and functional UNIX shell.

\section{Design Overview}
The design of my a1shell implementation is rather simple. Before the a1shell process begins
parsing commands from the stdin, it forks a child process (called a1monitor).
These two processes have the following design.
\begin{enumerate}
   \item a1shell
   \begin{itemize}
     \item this is the parent process of the main program
     \item scans for user input from stdout
     \item determines what the user command is via if else statements
     \item \textbf{cd}: uses chdir() system call to change user directory
     \item \textbf{pwd}: prints working directory to stdout with getpwd
     \item \textbf{umask}: prints current mask as well as S_IRWXG, S_IRWXO, and S_IRWXU
     \item \textbf{done}: terminates the a1shell process via the _exit() system call \\
     Note: as per the specifications in the assignment outline, I have designed the a1shell
     process to terminate without waiting on its child process. The child process (a1monitor) has
     been designed to terminate once its parent (a1shell) has terminated (see a1monitor design
     below).
     \item \textbf{cmd arg1 arg2 ...}: forks one child process which uses execl() to execute
     "/bin/bash cmd arg1 arg2 ...". The parent process waits for the child to terminate and
     prints the command execution time of the parent and child processes to stdout.
   \end{itemize}
   \item a1monitor
   \begin{itemize}
     \item this is the child process of the main program
     \item is in a while loop as long as its parent pid is the pid of a1shell
     \item the process prints the current time and date, followed by the information found
     in /sys/loadavg to stdout
     \item the process is designed to terminate AFTER its parent process terminates.
   \end{itemize}
\end{enumerate}

\section{Project Status}

\section{Testing and Results}
\textbf{Timing Test}:
in order to test the timing results of the execl process, I created a  bash script
"wait.sh" that simply sleeps for a specified amount of time.
\being{verbatim}
  #! /bin/bash

  sleep \$1
  return 0
\end{verbatim}
If i ran ./wait.sh 10 from a1shell, I would expect to see:
\being{verbatim}
  real time: 10000
\end{verbatim}


\section{Acknowledgments}
https://linux.die.net

\end{document}












