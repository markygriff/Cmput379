\documentclass{article}
\usepackage{graphicx}
\usepackage{bm}

\begin{document}

\title{CMPUT 379 Assignment 3}
\author{Mark Griffith - 1422270}

\maketitle

\section{Objective}
This assignment was an oppurtunity for myself to build experience
working with client-server development using TCP sockets for
communication.

\section{Design Overview}

\textbf{Program File:} a3chat.c \\
\textbf{Executable:} a3chat \\
\textbf{Function Returns:} 0 on success and -1 on failure. \\
\textbf{Makefile Options}: make $|$ make clean $|$ make tar \\

\noindent
\textbf{usage()}: Prints the proper syntax for calling the a1shell executable. \\

\noindent
\textbf{main()}: First, main() limits the CPU time of the program to avoid a hanging
process.
Next, the program either becomes a server or client program based
on the command line argument. \\

\noindent
\textit{Note: main() uses the return call to exit the program anytime BEFORE the processes
are created. This is purely a style choice and functions the same as \_exit()} \\

\noindent
\textbf{remove\_subsring()}: simple helper function to remove a substring. \\

\noindent
\textbf{server}: ./a3chat -s portnumber nclients \\
- Utilizes I/O multiplexing using poll()\\
- Uses a poll() timeout = 0 in order to accurately measure
the passage of real time.
- Uses an array of custom client structs to keep track
of client data and their chat sessions. \\
- Uses signal() to track when to print the activity report \\

\noindent
I designed the server such that if a read of write failure
occurs, the server remains active (does not terminate).
The user is instead notified that there was an issue with
the server. I feel this is a more realistic approach to
handling the read/write error because there is a good
chance the error had occured due to a faulty client
program, and so the server should remain in a stable state. \\

\noindent
\textbf{client}: ./a3chat -c portnumber server-ip \\
- Clients are controlled using two functions...

\begin{itemize}
  \item \textbf{ begin\_client() }:
  \item \textbf{ client\_chat() }:
\end{itemize}

\noindent
- Assumes if there is a server error, the response
message will begin with "[server] error:"\\
- Connects with server from within client\_chat() program\\
- The client that opens a chat session will always
belong to that chat session i.e. the client
who opened the session will recieve the messages
it sends to the other clients. \\

\noindent
\textit{Note: I chose to use a forked child process to poll
from stdin in both the server and client chat sessions. This
design choice was to keep code seperated and clean so to
prevent or isolate buggy behavior. As well, it allows the
user to exit either session seamlessly.} \\

\noindent
\textbf{ Some things to note: } \\
  - If there is no server present, the client cannot open a chat session. \\
  - The maximum buffer size is defined at the top of the file. Currently, it
    is set to 1024. \\
  - Creatin=on of the KAL messages is handled by the get\_kal\_msg() funciton. \\
  - There is a debug print macro that is activated by defining 'DEBUG'.
  - Activity report is set to display every 15 seconds. \\

\section{Project Status}
This project currently implements all functionalities outlined in the
assignment outline. \\

\noindent
Some difficulity in the implementation of this
assignment came from finding legitimate ways of testing the several
functionalites of the code. Fortunately, I found that testing gave me the
oppurtunity to learn and better my understanding of the various system calls
necessary for the completion of this assignment.
Other difficulties included finding a means of tracking clients
and their chat sessions. I came up with the idea of storing client
info inside a struct to simplify this issue.


\section{Testing and Results}

A major part of my testing and debugging was done with
the help of the DEBUG\_PRINT macro I created for
debug printing. It can be enabled by defining DEBUG. \\

\noindent
\textbf{Singular Client Session Testing}: \\
Testing of a single client on a server was done
relatively simply, with debug printing. I came up
with unusual input command sequences to attempt to
break the program and asserted that the program
continued to function normally.
Once I established that a single client session was
working properly, I moved on to testing multiple clients
on a single session. \\

\noindent
\textbf{Multi Client Session Testing}: \\
This is where testing became a little trickier.
I chose to use a terminal multiplexer (tmux) to help
run the a2chat program as multiple users on the same
server. The first tests done were basic functionality
tests such as sending messages, opening and closing
chat sessions, and extensively using the who and
to commands.\\

\noindent
Next, I wanted to cover some of the more obscure cases
that might not have been thought about during
development. Some examples of this included...

\begin{itemize}
  \item Opening maximum+1 number of users on server.
  \item  Adding a user to the chat session, closing that
    user, and sending a message to that user.
  \item Trying to open a chat session on a maxed out
    server, closing an open chat session, and trying
    to open the chat session again.
  \item Adding a user that does not exist to the chat
    session
\end{itemize}

\noindent
After I was satisfied with my results from testing
multiple clients on a session, I repeated my testing
for single client sessions to check that none of my
debugging of multi client sessions had broken the
program. \\

\section{Acknowledgments}
linux.die.net

\end{document}












