\documentclass{article}
\usepackage{graphicx}
\usepackage{bm}

\begin{document}

\title{CMPUT 379 Assignment 4}
\author{Mark Griffith - 1422270}

\maketitle

\section{Objective}
This assignment was an oppurtunity for myself to build experience
working with a conceptual virtual memory system. It also allowed me to
study and compare the efficiencies of several page fault handling strategies.

\section{Design Overview}

\textbf{Source Files:} a4vmsim.c pfault.c memory.c ptable.c options.c statistics.c \\
\textbf{Executable:} a4vmsim pagesize memsize strategy\\
\textbf{Makefile Options}: make $|$ make clean $|$ make tar \\

\noindent
\textbf{a4vmsim.c}: handles option parsing, program control flow, and vm simulation. \\

\noindent
\textbf{pfault.c}: defines page fault handler functions for the supported strategies
(none, random, second chance, lru) \\
- random: randomly chooses a page from physical memory array, evicts the page, and loads the new page
to that memory location. \\
- second chance: searches through physical memory array until the page has a chance of 0,
evicts the page, and loads the new page. \\
- lru: searches through the physical memor array until page with the minimum count is found,
evicts the page, and loads the new page. \\

\noindent
\textbf{memory.c}: controls and models the physical memory for the simulation. \\
- uses an array of page table entries to represent the physical memory (pmem). \\
- defines memory eviction and memory loading functionality. \\

\noindent
\textbf{ptable.c}: models the page table and page table entries for the simulation. \\
- defines the page table entry struct. \\
- handles page table generation. \\
- defines page table entry lookup functionality. \\

\noindent
\textbf{options.c}: stores the opts struct for global option parameters.

\noindent
\textbf{statistics.c}: handles simulation statistic are storing and printing. \\
- defines global stats struct to store simulation statistics. \\
- defines printing of final simulation output. \\

\noindent
\textbf{ Some things to note: } \\
- \\

\section{Project Status}
This project currently implements all functionalities outlined in the
assignment outline. \\

\noindent
The majority of the difficulity in the implementation of this
assignment came from initially understanding how virtual memory works and
more specifically, how to simulate it.
Fortunately, I found that this gave me the oppurtunity to learn and
better my understanding of virtual memory and how it is implemented. \\

\section{Testing and Results}

\noindent
Testing for this assingment was done in two parts.

\noindent
\textbf{1. Unit Testing:}
I first had to run some unit tests in order to prove that the program's individual
functions were working properly. I did this manually by piping pre-generated
reference strings (in a text file) into the program. These reference strings
were designed to target specific functions, of which I knew the expected output
behaviour. For simplicity, I created the reference strings in the format 0xXXXXXXXXX
and so I had to alter how the program was translating the input. \\

\noindent
For example, if I wanted to test one of the page fault strategies,
I would create reference strings that attempt to access the same memory location. \\

\noindent
\textbf{2. Stress Testing:}
Due to the wide range of input parameters the program is required to handle, I needed
to ensure it could properly handle as many combinations of those parameters as possible.
Stress testing felt like an appropriate way of dealing with this.
All I did for this was create a simple bash script that ran through many different input
parameters. Since the program's simulation processes 20000000 ref strings in roughly
30 seconds, I opted to stress test with a static number of generated reference strings -
i.e. the nref argument of the mrefgen script was held around 2000000. This was sufficient
becuase that number of ref strings would still trigger the virtual memory functionality
and it allowed for testing to be performed rather quickly. \\

\section{Acknowledgments}
linux.die.net
stackoverflow.com

\end{document}












